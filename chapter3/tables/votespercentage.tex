\begin{table}

\caption{\label{tab:votespercentage}Votes in favor of Laws}
\centering
\begin{tabular}[t]{llrrrrrr}
\toprule
\multicolumn{2}{c}{ } & \multicolumn{3}{c}{Eerste Kamer} & \multicolumn{3}{c}{Tweede Kamer} \\
\cmidrule(l{3pt}r{3pt}){3-5} \cmidrule(l{3pt}r{3pt}){6-8}
law & impact & confessional & liberal & socialist & confessional  & liberal  & socialist \\
\midrule
Inkomstenbelasting 1893 & 1.6\% - 2.65\% & 0.25 & 0.82 &  & 0.31 & 0.88 & 0.50\\
Inkomstenbelasting 1914 & 1.9\% - 3.55\% & 1.00 & 1.00 & 1.00 & 0.68 & 1.00 & 1.00\\
Staatsschuldwet 1914 & 0 & 1.00 & 1.00 & 1.00 & 0.00 & 0.11 & 0.91\\
Successiewet 1878 & 1\% & 0.17 & 0.67 &  & 0.29 & 0.92 & \\
Successiewet 1911 & 2\% & 1.00 & 1.00 &  & 0.86 & 1.00 & 1.00\\
Successiewet 1916 & 5\% & 0.00 & 0.94 & 1.00 & 0.17 & 1.00 & 1.00\\
Successiewet 1921 & 7\% & 0.75 & 0.90 & 1.00 & 0.74 & 0.50 & 1.00\\
\bottomrule
\multicolumn{8}{l}{\rule{0pt}{1em}\footnotesize{Percentage of upper house and lower house members having voted in favor of fiscal reforms.}}\\
\multicolumn{8}{l}{\rule{0pt}{1em}\footnotesize{Impact calculated as one-off (Successiewet) or yearly (Inkomstenbelasting) expected payments in the standard regime}}\\
\multicolumn{8}{l}{\rule{0pt}{1em}\footnotesize{with a wealth of 100,000 1900 guilders.}}\\
\end{tabular}
\end{table}
